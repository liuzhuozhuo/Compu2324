\documentclass[11pt, twoside]{article} %iniciación del documento de tipo artículo, con tamaño de letra 11pt.
\usepackage[a4paper,total={6in, 8in},left=25mm, asymmetric]{geometry} %Cambio en los bordes y margenes del documento

\usepackage{fancyhdr} %Paquete para organizar y añadir header con los nombres

\usepackage{amsmath}

\usepackage[spanish,es-tabla,es-nodecimaldot]{babel}
\usepackage{tabularx}
\usepackage{booktabs}

\usepackage{caption}

\usepackage{graphicx}
\usepackage[hidelinks]{hyperref}
\usepackage{macro}

\fancypagestyle{main}{
    \fancyhf{}
    \fancyhead[L]{\thepage}
    \fancyhead[RO]{ZhuoZhuo L.}
    \renewcommand{\headrulewidth}{.4pt}
    \setlength{\headheight}{52pt}
}

\pagestyle{empty}

\begin{document}

\begin{figure}[h!]
    \minipage{0.87\textwidth}
	\includegraphics[width=3cm]{Icons/ugr.jpg}
	\endminipage
    \minipage{0.87\textwidth}
	\includegraphics[height = 2.5cm, width=3cm]{Icons/facultad_ciencias.png}
	\endminipage
	%%\vspace{-1cm}
\end{figure}

\vspace{0.3cm}

\begin{center}
    \Huge \textbf{Física Computacional}\\
    		\vspace{0.4cm}
    \LARGE \textbf{Voluntario 1:}  
    Simulación con dinámica molecular de un gas con un potencial de Lennard-Jones
\end{center}

\vspace{1cm}

\vspace{1cm}

\begin{center}
    \large \textbf{Resumen}\\
    		\vspace{0.2cm}
    \normalsize
    En este informe tenemos como objetivo ...

\end{center}

\vspace{1cm}

\begin{flushright}
    \large Zhuo Zhuo Liu 
    \\
    \vspace{0.4cm}
    \textbf{Grado en Física}
\end{flushright}

\newpage

\setcounter{page}{0}
\tableofcontents
\newpage

\pagestyle{main}

\section{Introducción}

\section{Planteamiento del problema}

Para poder 

\subsection{Condiciones iniciales}
Al introducir las condiciones iniciales del sistema, debemos de tener cuidado de
no colocar 2 partículas muy cercas entre ellas inicialmente, ya que puede provocar


\subsection{Condiciones de contorno}

Para introducir la condición de contorno bidimensional periódica empleamos 
2 funciones, una para la posición de las partículas, y la otra para la 
distancia entre partículas.

Empecemos por la posición, para ello al introducir el vector posición 
debemos de comprobar que en caso de tener alguna coordenada mayor que L,
imponer la periodicidad, esto lo haremos usando el operador resto. En Python
definiríamos la siguiente función:

\begin{verbatim}
    def cond_contorno(r):
        return r%L
\end{verbatim}

donde r es un array de NumPy de dimensiones Nx2 donde se almacena la 
posición de todas las N partículas. 

\vspace{3mm}

Por otro lado para la distancia entre partículas, emplearemos la siguiente
lógica.
\begin{enumerate}
    \item Calculamos el vector $\vec{R}_{ij}$ que une 2 puntos ($r_j, r_i$).
         Dicho vector formará un ángulo $\theta$ con el eje x.
    \item A partir de dicho ángulo podemos calcular la distancia H (véase
        figura), si el módulo de $\vec{R}_{ij}$ supera la distancia H, sabemos 
        que no estamos calculando la distancia más corta entre los dos 
        puntos.
s    \item En dicho caso redefinimos el vector $\vec{R}_{ij}$, como:
\end{enumerate}

\begin{equation}
    \vec{R}_{ij} = - \vec{R}_{ij} \frac{H-|\vec{R}_{ij}|}{|\vec{R}_{ij}|}
    \label{eq:redef_Rij}
\end{equation}

Implementando en función tendría la siguiente forma:

\begin{verbatim}
    def compute_distance(r):
        R = np.zeros((N, N, 2))
        for i in range(1, N):
            for j in range(i, N):
                R[i, j] = r[j]- r[i]
                angle = np.arctan(R[i, j, 1]/R[i, j, 0])
                max_length = abs(L/np.cos(angle%(np.pi/4)))
                norm = np.linalg.norm(R[i, j])
                if(norm > max_length):
                    R[i, j] =  -R[i, j] *(max_length - norm) / norm
                R[j, i] = R[i, j]
        return R
\end{verbatim}

\subsection{Potencial Lennard-Jones}

Una vez tenido las funciones para imponer la condición de contorno. Podemos
calcular la fuerza ejercida entre partículas por el potencial de Lennard-Jones.

\begin{equation}
    V(r) = 4\epsilon\brackets{\parenthesis{\frac{\sigma}{R}}^{12}-
        \parenthesis{\frac{\sigma}{R}}^{12}}
    \label{eq:Lennard_Jones_potential}
\end{equation}

donde se ha usado $R$ en lugar de $r$, para coincidir en la notación empleada en 
el código.

\vspace{3mm}

Entonces la fuerza de interacción entre las partículas viene dado por:

\begin{equation}
    \vec{F}(\vec{R}) =  - 4\epsilon\brackets{6\parenthesis{\frac{\sigma}{R}}^{5}-
    12\parenthesis{\frac{\sigma}{R}}^{11}}
\end{equation}

Para calcular la aceleración de la partícula, se suma la fuerza de interacción con
todas las demás las partículas, y se divide por la masa.

Implementando en todo esto en la función que te devuelve la aceleración del sistema
en función de la posición de las partículas.

\begin{verbatim}
    def lennard_jones(r):
        R = compute_distance(r)
        acc = np.zeros(N, 2)
        for i in range(N):
            for j in range(N):
                if(i!=j):
                    norm  = np.linalg.norm(R[i, j])
                    acc[i] = 4*epsilon(6*(sigma/norm)**5-
                        -12*(sigma/norm)**12)*R[i, j]/(norm*m)
        return acc
\end{verbatim}




\newpage

\appendix

\section{Tabla de valores}


\newpage

\section{Análisis de errores}


\end{document}
