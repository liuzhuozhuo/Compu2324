\documentclass[11pt, twoside]{article} %iniciación del documento de tipo artículo, con tamaño de letra 11pt.
\usepackage[a4paper,total={6in, 8in},left=25mm, asymmetric]{geometry} %Cambio en los bordes y margenes del documento

\usepackage{fancyhdr} %Paquete para organizar y añadir header con los nombres

\usepackage{amsmath}

\usepackage[spanish,es-tabla,es-nodecimaldot]{babel}
\usepackage{tabularx}
\usepackage{booktabs}

\usepackage{caption}

\usepackage{graphicx}
\usepackage[hidelinks]{hyperref}
\usepackage{macro}

\fancypagestyle{main}{
    \fancyhf{}
    \fancyhead[L]{\thepage}
    \fancyhead[RO]{ZhuoZhuo L.}
    \renewcommand{\headrulewidth}{.4pt}
    \setlength{\headheight}{52pt}
}

\pagestyle{empty}

\begin{document}

\begin{figure}[h!]
    \minipage{0.87\textwidth}
	\includegraphics[width=3cm]{Icons/ugr.jpg}
	\endminipage
    \minipage{0.87\textwidth}
	\includegraphics[height = 2.5cm, width=3cm]{Icons/facultad_ciencias.png}
	\endminipage
	%%\vspace{-1cm}
\end{figure}

\vspace{0.3cm}

\begin{center}
    \Huge \textbf{Física Computacional}\\
    		\vspace{0.4cm}
    \LARGE \textbf{Voluntario 2:}  
    Estudio del péndulo doble con un algoritmo Runge-Kutta.
\end{center}

\vspace{1cm}

\vspace{1cm}

\begin{center}
    \large \textbf{Resumen}\\
    		\vspace{0.2cm}
    \normalsize
    En este informe tenemos como objetivo ...

\end{center}

\vspace{1cm}

\begin{flushright}
    \large Zhuo Zhuo Liu 
    \\
    \vspace{0.4cm}
    \textbf{Grado en Física}
\end{flushright}

\newpage

\setcounter{page}{0}
\tableofcontents
\newpage

\pagestyle{main}

\section{Introducción}

\section{Planteamiento del problema}
Primero debemos de hallar las expresiones de los momentos angulares
a partir del Lagrangiano del sistema. 

\begin{equation}
    \mathcal{L} = \dot{\phi}^2 + \dot{\phi}\dot{\psi} \cos(\psi - \phi) +
     \frac{1}{2}\dot{\psi}^2 - 2g(1-\cos\phi) - g(1-\cos\psi) = 
\end{equation}

Recordemos que para simplificar las ecuaciones de movimiento, 
hemos considerado que $\dot{\psi} = 0$

\begin{equation}
    \mathcal{L} = \dot{\phi}^2 - 2g(1-\cos\phi) - g(1-\cos\psi)
\end{equation}

\begin{equation}
    p_{\phi} = \frac{\partial L}{\partial \dot{\phi}} = 2\dot{\phi} + 
     \xrightarrow{} \dot{\phi} = \frac{p_{\phi}}{2}
\end{equation}

\begin{equation}
    p_{\psi} = \frac{\partial L}{\partial \dot{\psi}} = 0 
\end{equation}

Expresando la Hamiltoniana del sistema en términos de los momentos 
angulares, obtenemos:

\begin{equation}
    H = \frac{p^2_\phi}{4} + 2g(1 - \cos\phi) + g(1 - \cos\psi)
\end{equation}

\subsection{Ecuaciones del movimiento}


\newpage

\appendix

\section{Tabla de valores}


\newpage

\section{Análisis de errores}


\end{document}
