\documentclass[11pt, twoside]{article} %iniciación del documento de tipo artículo, con tamaño de letra 11pt.
\usepackage[a4paper,total={6in, 8in},left=25mm, asymmetric]{geometry} %Cambio en los bordes y margenes del documento

\usepackage{fancyhdr} %Paquete para organizar y añadir header con los nombres

\usepackage{amsmath}

\usepackage[spanish,es-tabla,es-nodecimaldot]{babel}
\usepackage{tabularx}
\usepackage{booktabs}

\usepackage{caption}

\usepackage{graphicx}
\usepackage[hidelinks]{hyperref}
\usepackage{macro}

\fancypagestyle{main}{
    \fancyhf{}
    \fancyhead[L]{\thepage}
    \fancyhead[RO]{ZhuoZhuo L.}
    \renewcommand{\headrulewidth}{.4pt}
    \setlength{\headheight}{52pt}
}

\pagestyle{empty}

\begin{document}

\begin{figure}[h!]
    \minipage{0.87\textwidth}
	\includegraphics[width=3cm]{Icons/ugr.jpg}
	\endminipage
    \minipage{0.87\textwidth}
	\includegraphics[height = 2.5cm, width=3cm]{Icons/facultad_ciencias.png}
	\endminipage
	%%\vspace{-1cm}
\end{figure}

\vspace{0.3cm}

\begin{center}
    \Huge \textbf{Física Computacional}\\
    		\vspace{0.4cm}
    \LARGE \textbf{Voluntario 2:}  
    Estudio del péndulo doble con un algoritmo Runge-Kutta.
\end{center}

\vspace{1cm}

\vspace{1cm}

\begin{center}
    \large \textbf{Resumen}\\
    		\vspace{0.2cm}
    \normalsize
    En este informe tenemos como objetivo ...

\end{center}

\vspace{1cm}

\begin{flushright}
    \large Zhuo Zhuo Liu 
    \\
    \vspace{0.4cm}
    \textbf{Grado en Física}
\end{flushright}

\newpage

\setcounter{page}{0}
\tableofcontents
\newpage

\pagestyle{main}

\section{Introducción}

\section{Planteamiento del problema}
\subsection{Ecuaciones del movimiento}
Primero debemos de hallar las expresiones de los momentos angulares
a partir del Lagrangiano del sistema. 

\begin{equation}
    \mathcal{L} = \dot{\phi}^2 + \dot{\phi}\dot{\psi} \cos(\psi - \phi) +
     \frac{1}{2}\dot{\psi}^2 - 2g(1-\cos\phi) - g(1-\cos\psi) 
\end{equation}

Hallamos las expresiones del momento angular $p_\phi$ y $p_\psi$, en función
de las velocidades angulares $\dot{\phi}$ y $\dot{\psi}$ através de las 
parciales de $\mathcal{L}$ con respecto a las velocidades angulares.

\begin{equation}
    p_\phi = \frac{\partial \mathcal{L}}{\partial \dot{\phi}} = 2\dot{\phi} + \dot{\psi}\cos(\psi - \phi)
\end{equation}

\begin{equation}
    p_\psi = \frac{\partial \mathcal{L}}{\partial \dot{\psi}} = \dot{\psi} + \dot{\phi}\cos(\psi - \phi)
\end{equation}

Despejando las velocidades en función del momento, permite expresar el 
Hamiltoniano en función de dichos momentos.

\begin{equation}
    \dot{\phi} = \frac{p_\phi - p_\psi\cos(\psi - \phi)}{2-\cos^2(\psi - \phi)}, \quad 
    \dot{\psi} = \frac{2p_\psi - p_\phi\cos(\psi - \phi)}{2-\cos^2(\psi - \phi)}
\end{equation}

\begin{equation}
    \begin{split}
        H &= \dot{\phi}^2 + \dot{\phi}\dot{\psi} \cos(\psi - \phi) +
\frac{1}{2}\dot{\psi}^2 + 2g(1-\cos\phi) + g(1-\cos\psi)  \\
 &=\frac{p_\phi^2 + p_\psi^2 - 2p_\phi p_\psi \cos(\psi - \phi)}{2-\cos^2(\psi - \phi)} + 2g(1-\cos\phi) + g(1-\cos\psi)
    \end{split}
\end{equation}

De donde se obtiene las ecuaciones de movimiento a partir de las 
ecuaciones de Hamilton.

\begin{equation}
        \dot{\phi} = \frac{\partial H}{\partial p_\phi} = \frac{p_\phi - p_\psi\cos(\psi - \phi)}{2-\cos^2(\psi - \phi)} 
\end{equation}

\begin{equation}
    \dot{\psi} = \frac{\partial H}{\partial p_\psi} = \frac{2p_\psi - p_\phi\cos(\psi - \phi)}{2-\cos^2(\psi - \phi)}
\end{equation}

\begin{equation}
    \dot{p_\phi} = -\frac{\partial H}{\partial \phi} =  \frac{p_\phi p_\psi \cos^2(\psi - \phi) - (2p_\psi^2 + p_\phi^2)\cos(\psi - \phi) + 2p_\phi p_\psi }{(2-\cos^2(\psi - \phi))^2}2\sin(\phi - \psi) -2g\sin\phi
\end{equation}

\begin{equation}
    \dot{p_\psi} = -\frac{\partial H}{\partial \psi} =  \frac{p_\phi p_\psi \cos^2(\psi - \phi) - (2p_\psi^2 + p_\phi^2)\cos(\psi - \phi) + 2p_\phi p_\psi }{(2-\cos^2(\psi - \phi))^2}2\sin(\psi - \phi) -g\sin\psi
\end{equation}

\subsection{Condiciones iniciales}



\newpage

\appendix

\section{Tabla de valores}


\newpage

\section{Análisis de errores}


\end{document}
